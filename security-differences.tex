Speaking about a closed-source developer, Stallman says "if he puts in a
malicious feature, the users can't fix it" \citeyear[para. 36]{rms2011}.
One can reasonably assume that some closed-source developers do not try
to introduce malicious features. But any developer, closed-source or
open source,can still accidentally introduce security
vulnerabilities. Disregarding Stallman's position on surveillance present in
proprietary software, when one examines both open source and closed-source
software, neither has a clear security advantage over the other.

With open source software, users can more easily find security holes because
they have access to source code, and user can propose fixes immediately.
With closed-source software, when users notice a security flaw, they can report
it to the developers, but the user has to rely on the developers to implement a
fix \cite[para. 17]{kadura}.

Because closed-source software goes through a tremendous amount of testing
by the group maintaining the project, the speed of patches and fixes is limited
by the size of the development team. This becomes more of an issue as projects
get larger; the developers have much more code to go through in order to fix the
bug. With open source however, as the project grows, so does the number of
people working on the code review process, as some users actively try to help
fix the software \cite[p. ~245]{boulanger}.

Kadura and Schryen compare security vulnerabilities in two products, Microsoft
Office and OpenOffice, with many users. They only count vulnerabilities listed
as "Common Vulnerabilities and Exposures (CVE) entries by the CVE editorial
board, which was itself created by the MITRE corporation"
\citeyear{kadura}. They found that Microsoft Office had a total of 108
vulnerabilities, while OpenOffice had only 16. Both Microsoft Office and
OpenOffice had comparable numbers of vulnerabilities listed as low severity,
three and two respectively, but Microsoft Office had roughly 7 times as many
medium and high severity vulnerabilities as OpenOffice. The authors recognize
that the number of medium and high severity vulnerabilities for OpenOffice may
be substantially lower because patches for vulnerabilities were applied before
the CVE editorial board tallied the numbers \cite{kadura}.

Many security vulnerabilities do not depend at all on viewing the source code.
In an attempt to gain access to user data, black-hat hackers can use a program
that tries millions of passwords until one works. Many programs have ways of
detecting this, so black-hats often employ  more clever approaches. With methods
such as SQL injection and buffer overflow, one can find vulnerabilities by
providing strange input to a program. If the program cannot handle the input,
the program may have unexpected behavior that may give the hacker access he
should not have.\cite[p. ~8-9]{clarke}.

Even if a program can check for brute force attacks, or is designed to handle
any format of input without failing, its developers can never account for
a corrupt developer or administrator that works from the inside to take
advantage of users' data. Humans are often the weakest link in software
security. Otherwise, phishing, pharming, and other similar scams would not be
effective.
