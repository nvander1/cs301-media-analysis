Stallman maintains that a free society can only come about through the adoption
of free software, stating that he launched the GNU operating system for the
express purpose of freedom and not for technical innovation \citeyear[para.
48]{rms2011}. Stallman criticizes those who associate the free software
movement with the ilk of Linus Torvalds and others who would seek to shift the
focus of the free software movement away from ethics and towards practicality.
However, if Stallman truly wishes to see widespread adoption of free software,
then it would behoove him to also encourage others to capitalize on the
practical benefits that made Linux and other open source software solutions so
successful.

While the transparency of source code serves to inhibit the existence of
unethical and malicious code, it also serves to enhance the collaborative nature
of software development. A key feature of open source is that it creates an
environment in which users mutually borrow and contribute back to one another's
projects. The community that formed around the development of Linux embodies
this idea very closely. Jim Zemlin, the executive director of the Linux
Foundation, provides an illustration in which changes made to Linux to conserve
power on a cellphone can be used to benefit supercomputers that also run Linux
\citeyear[11:34]{zemlin}. Similarly, the increased availability of code allows
developers to adapt software to new challenges. By inheriting httpd's Common
Gateway Interface standard, the development of the Apache web server was
expedited, as it's creators no longer had the need to build these components
from scratch \cite[p. 7]{bisson}.

The overall restrictiveness of proprietary software presents a large
disadvantage to its users.  Because software is built from source code, the
support for any piece of software is ultimately tied to the ability to edit and
view that source code. Having access to a product's source code decreases
dependency on that product, which is critical should that product become
discontinued or should its vendor declare bankruptcy.  Thus, vendor lock-in
threatens software users with eventual obsolescence, even in cases where it was
possible to retroactively obtain the rights to proprietary code . In one such
case involving the migration to new electronic voting machines in New York, the
code obtained to make these machines function was licensed for testing and not
for deployment \cite[p. 916]{colannino}. Moreover, this issue could have been
    avoided had the code been released with a free software license, such as the
    GNU public license. Open source licenses protect the continuous improvements
    made on software that prevent vendor lock-in \cite[p. 919]{colannino}.

Open source software also has a unique quality assurance mechanism in the form
of its peer-review process. The communities that develop around open source
software are comprised of passionate individuals who are devoted to problem
solving \cite[p. 19]{bisson}. These users endlessly patch vulnerabilities and
provide new features. In 2013, it was determined that 6,782 lines of code are
added and subtracted to Linux on a daily basis (Zemlin, 2013, 12:03).
Consequently, the  constant revision of Linux ensures its stability and
relevance. Yet, the process of open source peer-review is not as straightforward
as some would believe. In a study analyzing the bug reports for Mozilla Firefox,
Wang et. al. reevaluates “Linus's Law,” or the idea that “given enough eyeballs,
all bugs are shallow”\cite[p. 52]{wang}.  They determine that Mozilla's
peer-review process lacked a central focus due to the differing levels skill
levels of bug reporters, as disputes commonly arose over the relative importance
of certain bugs \citeyear[p. 52]{wang}. However, Wang et. al.  claim that
careful labeling of duplicate bug reports would solve this problem \citeyear[p.
52]{wang}.

Due to the ease and availability of open source software, one can easily match
a software need with an open source solution. When developing Facebook,
Mark Zuckerberg used Linux and other free software to build the word's most
successful social media platform \cite[6:18]{zemlin}. One can even argue that
a software need—not an ethical need—provided Stallman with the impetus to launch
the free software movement. Stallman's animosity towards proprietary software
originated from a printer's software's lack of extensibility. When denied
permission to add features to the software, Stallman vowed that he
would never let a programmer share in his frustration \citeyear[para.
19]{rms2015}.
