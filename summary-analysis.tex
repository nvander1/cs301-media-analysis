In a talk given to university students in Paris, Richard Stallman warns of being a prisoner of the digital world. It does not seem likely that Stallman, a computer scientist, would preach against a digital lifestyle. However, Stallman takes issue not with computers and software on their own, but rather with software and services that disrespect their users' freedom. He believes proprietary software allows companies to insert malicious code that can spy on users. Companies that make this kind of software, he says, not only deny users freedom of privacy, but also establish control over those users by keeping their source code secret. To combat this loss of control, Stallman argues “we have to extract people from digital society if it doesn't respect their freedom or we have to make it respect their freedom” \citeyear[para. 2]{rms2011}.

Early in the talk, Stallman clarifies what he means by free software: software that respects users' freedom. Non-free software, he says, has potential to control its users. As an example, he points to computers running Microsoft Windows. Those machines do not respect users' freedom, he says, because they track "data about the use of the computer" \citeyear[para. 4]{rms2011}. Stallman adds that malicious software need not be on users' hardware; an online service can also spy on its users and control users' data . He points out a privacy double standard—companies do not protect users' privacy, but in the case of DRM, take great lengths to protect their own.

Stallman chiefly supports his argument for “digital extraction” with an analysis of what he believes to be the main threats to digital freedom: surveillance, censorship, restricted data formats, proprietary software, Internet services, and computer voting. While Stallman alludes to the more political topics of surveillance, censorship by governments and computer voting, he devotes much of his talk to the regulation and distribution of digital media and software.  Stallman envisions a world where users distribute content easily and freely; accordingly, he views any copy protection measures, such as the DMCA, as a form of censorship \citeyear[para. 96]{rms2011}. Stallman encourages his audience to fight the “digital handcuffs” imposed by copy protection that seek to control how they use content \citeyear[para. 98]{rms2011}. Stallman similarly opposes proprietary software. Stallman supports his claim for prohibiting proprietary software by citing its negative effects on education. He further argues society as a whole would benefit by severing its dependence on proprietary software \citeyear[para. 53]{rms2011}. To realize his vision of a world with predominantly free software, Stallman launched the GNU Project and the Free Software movement.

Stallman supports his arguments with examples carefully tailored to his audience. Because he delivered his talk at a French university, Stallman illustrates threats to digital freedom using examples from France and greater Europe. For instance, he mentions the tracking of bicycles in Paris to demonstrate threats posed by digital surveillance \citeyear[para. 8]{rms2011}, and he mentions secret file formats employed by Italian public television to demonstrate threats posed by restricted data formats \citeyear[para. 30]{rms2011}. Stallman also enhances his credibility through his use of French terms2011. Stallman's explanation of the meaning of “free”—or rather “libre”—software compliments the subtleties of French \citeyear[para. 34]{rms2011}. Stallman's extensive explanation of free software also lays an important foundation for the rest of his talk that can be understood by different audiences.

One should note that Stallman presents his views in a purely ethical context. In fact, Stallman believes one should only mention free software as “an ethical issue.” \citeyear[para. 63]{rms2011}. Stallman claims additionally that those who coined the term “open source” did so to avoid discussing the ethics of free software \citeyear[para. 52]{rms2011}. Stallman also refuses to entertain any arguments on the economics of free software, choosing to focus solely on ethics \citeyear[para. 34]{rms2011}.

When talking about a digital society, Stallman draws his ethical truths about a digital society from a non-digital society. For example, in his section entitled “the war on sharing,” Stallman, who views book lending as an “important social act,” criticizes the Amazon kindle for its inability to lend books \citeyear[para. 98]{rms2011}. Stallman maintains that residents of a digital society must fight to maintain the basic freedoms and rights enjoyed by the non-digital society of the past.

