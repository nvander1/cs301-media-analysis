Stallman \citeyear{rms2011} explains the need for free software in his lecture, but does not explain the economic power of free software.
Although free software is by definition free of charge, that does not mean that free software has no place in a capitalistic society.
Major companies contribute to open source software, and some are even core contributors and maintainers.
Using open source enables these companies to leverage the knowledge of the community, which is not possible when creating closed source software.
Open source software is also useful to companies in its finished form, for example, Apache web server ran two-thirds of major web sites \cite[p.~696]{powell}

Open source software can also become the basis for entrepreneurs starting new companies, as liberal software licenses do not restrict any kind of usage, including commercial.
One such example is OpenSimulator, which is a BSD licensed 3D software system.
Because the BSD license is commercially friendly, a startup can use OpenSimulator in their own software, and redistribute it \cite[p.~477-8]{yetis}.

OpenSimulator's community was the subject of a study on open entrepreneurship, examining how networking in an open source community can help an entrepreneur fulfill business goals.
The study ultimately found that entrepreneurs rely on the networks they form by actively participating in the open source community when starting new firms.
The study also found that entrepreneurs make open source communities better with their political skills while still achieving business goals \cite{yetis}; sharing environments do not exclude business opportunities, in fact, these communities greatly benefit capitalism.
An economy where costs to start a business are low can foster innovative technologies and increase the standard of living for everyone, since in-house software can be costly, open source can lower costs for business ventures.

Open source businesses have proved sustainable, Red Hat being an excellent example.
Red Hat makes money from subscriptions, training, and services instead of by licensing their software \cite[p.~31]{redhat}.
Red Hat's software is licensed under the GNU GPL, and also promises to not enforce patent rights on their code \cite[p.~65]{redhat}, making their software freely usable and distributable by anyone.
CentOS, for example, is a free clone of Red Hat Enterprise Linux, making the power of Red Hat software available to anyone who can download an installer.
Despite the existence of a free clone, Red Hat still makes a large amount of money, \$524 Million of revenue in Q3 FY 2016, up from \$456 Million in Q3 FY 2015 \cite[p.~24]{redhat}.
Red Hat is living proof that developing open source software can still generate revenue, which contributes to the economy by providing valuable services and advancing global technology.

When companies can make open source software profitable, the nature of open source hugely benefits the overall economy.
Open source technology enables a more efficient economy, because individuals and firms do not have to go through the laborious process of building everything from scratch, or bear the expense of paying large licensing fees to get anywhere with software development.
However, open sourcing a piece of software is not always the best solution as it could mean sacrificing profits.
One paper studied the economic viability of open sourcing a piece of software at a certain point.
The paper concludes that it makes sense to open source software that has high costs to maintain quality, because open source contributors can reduce the cost of maintaining that software in-house.
However, the costs to switch to open source must also be considered before making the switch \cite{caulkins}.
One example cited in the paper is the Doom-engine, which was very high quality software when it was released, but competitors eventually caused the cost of keeping the engine up to date to be too much for id Software.
Id Software open sourced the engine in 1997, but still made money by selling content that runs on the engine \cite[p.~1188]{caulkins}.

Although Stallman's lecture can give an almost Marxist view of software, open source still has a big place in a capitalistic society.
Much like global trade, sharing knowledge can benefit all parties involved and improve the overall efficiency of the economy.
Wasted resources, namely human resources, decrease the efficiency of the economy, so reducing the amount of duplicated work in software development is good for the overall economy.
