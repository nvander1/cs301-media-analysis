Richard Stallman's ideas, although radical, have merit.
Open source has many practical advantages, including the ability to build off one another's work without worrying about intellectual property lawsuits.
Proprietary software is generally hard to decode to learn how it works, and even if one was successful, publishing information about the internal workings could be considered a breach of license.
With open source, users gain the power to fix problems and create their own flavors to suit their specific needs, without having to contact the original publisher.
The open source community is a powerful entity, with brilliant problem solvers collaborating in an open setting, and this helps progress technology.

The open source community also is a place where entrepreneurs can network and eventually form new technology companies based on open source technologies.
By lowering the entry costs, innovation is bound to appear, contributing to the economy and raising the standard of living.
Existing open source businesses provide great value to other firms and individuals, since their well maintained software can be used by anyone, and company revenue improves the quality of the software.
Companies with legacy software can also benefit from open source, by releasing their code they can gain the help of the open source community, lowering the cost to maintain quality.

Along with quality, the open source community is also great for finding security vulnerabilities in software, as there's a large collection of people and firms who are willing to find and fix security flaws.
With access to source code, the users are actually able to have confidence that their software is operating in a certain way, because they can simply pull up the source code.
Lowering the amount of security holes in software helps fight cybercrime, and lead to a safer computing experience for all parties involved.

Stallman's ideas for a free and open source software ecosystem can be applied to an extent, bringing collaboration into software development and promoting idea sharing.
With free and open source software, software developers can easily build off each other and avoid repeating work and wasting time building things from scratch.
